%%% nuclearSlice.tex --- 
%% Author: snapp@jeeves.local
%% Version: $Id: nuclearSlice.tex,v 0.0 2008/06/01 03:14:09 snapp Exp$

\documentclass[10pt]{article}
\usepackage[width=6.75in]{geometry}
\usepackage{alltt}

\title{Guide to nuclearSlice, Version 0.1}
\date{\today}

\author{Robert R. Snapp\\
Department of Computer Science\\
University of Vermont\\
Burlington, VT 05405 U.S.A.\\
\texttt{snapp@cs.uvm.edu}}

\newcommand{\LSM}{\textsc{lsm}}

\begin{document}
\maketitle
%\begin{abstract} \end{abstract}

\setcounter{section}{-1}

\section{Warning}
The software in the \texttt{nuclearShape} project is still being developed and tested, and is not intended for public release.
It has been placed in the \texttt{github} repository to facilitate further development.
More information about this project will appear at \texttt{http://www.cs.uvm.edu/$\sim$snapp/nuclearShape}.


\section{Description}
Program \texttt{nuclearSlice} analyzes the geometry of cell nuclei
contained in one or more files in Zeiss's \LSM\ format, in either
an interactive or a batch mode. The program should be invoked from
a unix shell like the Terminal application in Mac OSX.

Syntax:

\begin{alltt}
nuclearSlice [options] <filename>
\end{alltt}

\noindent
where \texttt{<filename>} is either a path to a single \LSM\ file,
or a roster of \LSM files, described below. The command line options
are grouped in several categories. 

\begin{description}
\item{\tt -b <int>} Sets the band index to the integer argument.
\item{\tt -c <int>} Sets the column of the subregion to the provided integer argument.
\item{\tt -d <str>} Sets the directory path for input \LSM\ files.
\item{\tt -r <int>} Sets the row of the rectangular subregion to the provided integer argument.
\item{\tt -h <int>} Sets the height (number of rows) of the rectangular subregion to the integer
argument.
\item{\tt -l <int>} Sets the layer index to the provided integer argument.
\item{\tt -o <str>} Prints the output to the file inticated by the argument. If none is provided,
then the output is directed to \texttt{stdout}.
\item{\tt -v <int>} Enables verbose mode, for debugging, and sets the message level to the indicated value.
\item{\tt -w <int>} Sets the width of the rectangular subregion to the indicated value.
\item{\tt -G} Graphics (i.e., interactive) mode: activates a graphical browser for viewing the images
contained in the input \LSM\ file or files. (Disbaled by default.)
\item{\tt -H} Help mode: display the command line syntax.
\item{\tt -P} Activates projection mode: intensities of all available layers are summed together. (Disbled by default.)
\end{description}


\section{Interactive Mode}

To analyze an \LSM\ file under interactive mode, issue the command with the \texttt{-G} option.
Thus, to analyze a single file, enter the following command into a terminal shell:
\begin{center}
\begin{alltt}
nuclearSlice -G file.lsm
\end{alltt}
\end{center}
which loads the \LSM\ file named \texttt{file.lsm}, and displays the first layer (layer = 0), in a window.

Alternatively, if one desires to analyze a set of lsm files, then use a text editor (or other means)
to create a \emph{file roster}, which contains a single \LSM\ file name on each line:

\begin{center}
\begin{alltt}
# my file roster.
# Note that any line that begins with a pound sign (#) is treated as a comment
# and is ignored.
file1.lsm
file2.lsm
\(\qquad\vdots\)
file99.lsm
\end{alltt}
\end{center}

Additional parameters can be provided for each file in the file roster. Fields should
be assigned values in columns as shown:
\begin{center}
\begin{alltt}
# filename   col   row     width    height   threshold  
file1.lsm    128   128      256      256       42       
file2.lsm    167    89      108      139       35       
file2.lsm    318   243       98      156       42       
\(\qquad\vdots\)
\end{alltt}
\end{center}
Note that a file roster is a convenient way to analyze multiple blobs that belong to the same \LSM\ file.

In interactive mode, some of the keys on the keyboard are mapped to perform various functions.
Normally, one might enable projection mode (by pressing {\tt Shift-P}). Then view the set of pixels
that have intensities above the current threshold by pressing {\tt F2}. (Note on some systems the function
keys are mapped to system controls. In this case it may be possible to press {\tt fn-F2}, etc.)
The threshold can be varied by pressing the up and down arrow keys, or the page-up and page-down keys.
Once a threshold has been found that adequately fills in the interior of the largest blob, press {\tt m}
to apply a morphological closure operation. (The latter, fills in any isolated pixels inside the blobs
that lie below threshold.) Then press {\tt c} to assign colors to the largest components. The largest component
will be rendered red. The components can be viewed by pressing {\tt F5}. Then press {\tt h} to
construct the convex hull of the red component.

If one is more interested in a blob that is not the largest one, then one can crop the image around the blob
of interest using the {\tt r}, {\tt x} keys,
and the mouse.

\subsection{Keyboard Command Summary}

\begin{description}
\item{\tt 0} select and display band 0.
\item{\tt 1} select and display band 1.
\item{\tt a} prompts the user for an integer roster file offset. The value should be typed  in
the active terminal shell, followed by a carriage return.
\item{\tt c} assign a colored label to each connected component.
\item{\tt Shift-F} enable the user to enter a new filename in the active terminal shell.
\item{\tt h} construct the convex hull of the largest (red-labeled) component. The ratio of the area enclosed
by the convex hull to the area contained in the largest component is also computed and displayed.
\item{\tt i} toggles the visibility of the text information displayed in the image window.
\item{tt m} applies a morphological closure operation to the thresholded intensities. 
\item{\tt n} displays the next layer in the current image.
\item{\tt p} displays the previous layer in the current image.
\item{\tt Shift-P} toggles projection mode. (In projection mode only the sum of the
\item{\tt r} Begin and interactive region selection mode, which can be used to crop the image. After pressing {\tt r},
move the graphical cursor with one's mouse or pointing tool to the upper-left corner of the region of interest.
Then click and drag the mouse to the opposite (lower-right) corner, and then release. Pressing {\tt x} then extracts
(or crops) the image to this subregion. The coordinates of the subregion are displayed in the active terminal
for reference.
\item{\tt x} extract the chosen subregion (see command {\tt r}, above).
\item{\tt >} advance to the next file in the file roster.
\item{\tt <} go back to the previous file in the file roster.
\item{\tt F1} displays the image intensities.
\item{\tt F2} displays the pixels with intensity above the current threshold.
\item{\tt F3} applies a dialation to the thresholded image, and displays the dialation.
\item{\tt F4} displays the closure of the thresholed image.
\item{\tt F5} displays the labeled components of the closure.
\item{$\leftarrow$} displays the previous band or channel. For example, band = 0 might correspond to
intensities corresponding to different fluorescent dyes.
\item{$\rightarrow$} displays the next band or channel.
\item{$\uparrow$} increases the threshold value by 1.
\item{$\downarrow$} decreases the threshold value by 1.
\item{Page-Up} (or possibly fn-$\uparrow$), increases the threshold value by 5.
\item{Page-Down} (or possibly fn-$\downarrow$), decreases the threshold value by 5.
\end{description} 

\section{Batch mode}

This application can also be executed in batch mode in a terminal shell, by not invoking the \texttt{-G}
option. Thus,
\begin{alltt}
nuclearSlice -P -o lsmFiles.out  fileroster.txt
\end{alltt}
analyzes the \LSM\ files listed in fileroster.txt as a batch, and records the output in file lsmFiles.out.

\end{document}

%%% Local Variables: 
%%% mode: latex
%%% TeX-master: t
%%% End: 
